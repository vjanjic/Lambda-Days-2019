\documentclass{article}

\usepackage{amsmath}
\usepackage{amssymb}
\usepackage{semantic}
\usepackage{bbm}
\usepackage{xspace}

\usepackage{nag}

\newcommand{\s}[0]{\bar{s}}
\newcommand{\type}[3]{#1: #2\; #3}
\newcommand{\ing}[1]{#1 \in{} \Gamma}

\newcommand{\seq}[3]{#1 & \vdash{} & #2 & = & #3}
\newcommand{\rewrite}[3]{#1 & \vdash{} & #2 & \mapsto{} & #3}
\newcommand{\Rewrite}[3]{#1 & \vdash{} & #2 & \mapsto{}^{\!\!\!*} & #3}

\newcommand{\erlang}[0]{\textsc{Erlang}}
\newcommand{\skel}[0]{Skel}
\newcommand{\ie}[0]{i.e.\@\xspace}

\begin{document}

\section{Environment Definitions}

\[
  \begin{array}{l}
    \gamma = \text{environment} \\
    \mathbb{F} = \text{the set of functions in}\; \rho \\
    \mathbb{S} = \{\bar{f}, \mathrm{pipe}(s_1, s_2), \mathrm{farm}(s, n)\} \\
    \Gamma = \gamma \times \mathbb{S} \\
    \mathbb{L} = \text{set of all lists} \\
  \end{array}
\]

\section{Skeleton Definitions}

\[
  \begin{array}{rclcl}
    \Gamma & \vdash{} & \mathcal{F} & = & \{F, \;\mathtt{fun\; ?MODULE:}f\mathtt{/1}, \;\mathtt{fun}(x)
    \rightarrow \;\dots \;\mathtt{end}\} \\
    \bar{s},xs \in \Gamma, xs \in \mathbb{L} & \vdash & \mathrm{skel}(\bar{s}, xs) & = & \mathtt{skel:do}(\bar{s}, xs) \\[2em]
    f \in \Gamma & \vdash & \bar{f} & = & \{\mathtt{func},\; \mathcal{F}\} \\[2em]
    \bar{s}_1,\bar{s}_2 \in \Gamma & \vdash & pipe(\bar{s}_1,\dots,\bar{s}_n) & = & \{\mathtt{pipe}, [\bar{s}_1,\dots,\bar{s}_n]\} \\[2em]
    \bar{s} \in \Gamma, n \in \mathbb{N}^{+} & \vdash & farm(\bar{s}, n) & = & \{\mathtt{farm}, \bar{s}, n\} \\[2em]
    \bar{s}_c,\bar{s}_g \in \Gamma & \vdash & farm'(\bar{s}_c, \bar{s}_g) & = &
                                                               \{\mathtt{hyb\textunderscore
                                                               farm}, \bar{s}_c,
                                                               \bar{s}_g\} \\[1em]
    \begin{array}{r}
      \bar{s}_c,\bar{s}_g \in \Gamma, \\ 
      n,m \in \mathbb{N}^{0}, \\
      n + m > 0
    \end{array} \Bigg\} & \vdash & farm''(\bar{s}_c, \bar{s}_g, n, m) & = &
                                                               \{\mathtt{hyb\textunderscore
                                                               farm}, \bar{s}_c,
                                                               \bar{s}_g, n, m\} \\[2em]
    \bar{s}, p, c \in \Gamma & \vdash & cluster(\bar{s},p,c) & = & \{\mathtt{cluster}, \bar{s},
                                                                                 \mathcal{F}_p,
                                                                                 \mathcal{F}_c\}
    \\[.8em]
    \bar{s} \in \Gamma & \vdash & cluster(\bar{s}) & = &
                                                         \begin{array}{l}
                                                           \{\mathtt{cluster}, \bar{s}, \\
                                                           \quad \mathtt{fun\;
                                                           skel:partition/1},\; \\
                                                           \quad \mathtt{fun\; skel:combine/1}\}
                                                         \end{array} \\[.8em]
    \begin{array}{r}
      \bar{s}_c, \bar{s}_g \in \Gamma, \\
      n \in \mathbb{N}^{+}, \\
      m \in \mathbb{R}^{+}, \\
      x,y \in \mathbb{N}^{0}
    \end{array} \Bigg\} & \vdash & cluster'(\bar{s}_c, \bar{s}_g, n,
                               m, x, y) & = & \{\mathtt{hyb\textunderscore
                                              cluster}, \bar{s}_c, \bar{s}_g, n,
                                              m, x, y\} \\[2.5em]
    \begin{array}{r}
      \bar{s}_c, \bar{s}_g, g, h, i \in \Gamma, \\
      n \in \mathbb{R}^{+}
    \end{array} \Big\} & \vdash & cluster''(\bar{s}_c, \bar{s}_g, n,
                              g, h, i) & = & \{\mathtt{hyb\textunderscore cluster}, \bar{s}_c, \bar{s}_g, n,
                                               \mathcal{F}_g, \mathcal{F}_h,
                                             \mathcal{F}_i\} \\
    xs, l \in \Gamma, l \in \mathbb{N}^{+}, xs \in \mathbb{L} & \vdash & chunk(xs, l) & = & \mathtt{skel:chunk}(xs, l) \\
  \end{array}
\]

\section{Rewrite Rules}

\[
  \begin{array}{rclcl}
    % \Gamma, f \in \mathbb{F} & \vdash & f & \rightarrow & \bar{f} \\
    \Gamma, n \in \mathbb{N}^{+} & \vdash & \bar{f} & \rightarrow & farm(\bar{f}, n) \\
    % \Gamma & \vdash & g(f(x)) & \rightarrow & \mathrm{skel}(\mathrm{pipe}(\bar{f}, \bar{g}), x) \\
    % \Gamma & \vdash & [f_1(f_2(\dots f_n(x)\dots)) || x \leftarrow xs] &
    %                                                                      \rightarrow
    %                                                               &
    %                                                                 \mathrm{skel}(\mathrm{\overline{pipe}}([\bar{f}_n,
    %                                                                 \dots, \bar{f}_2, \bar{f}_1]), xs)
    % \\
    \Gamma, \bar{s}_{g} \in \mathbb{S} & \vdash & \bar{f} & \rightarrow & farm'(\bar{f}, \bar{s}_{g}) \\
    \Gamma, \bar{s}_{g} \in \mathbb{S} & \vdash & farm(\bar{s}_c, n) & \rightarrow & farm'(\bar{s}_c, \bar{s}_{g}) \\
    \Gamma, \bar{s}_{g} \in \mathbb{S}, n,m \in \mathbb{N}^{0} & \vdash & \bar{f} & \rightarrow & farm''(\bar{f}, \bar{s}_{g}, n, m) \\
    \Gamma, \bar{s}_{g} \in \mathbb{S}, m \in \mathbb{N}^{0} & \vdash &
                                                                        farm(\bar{s}_c, n) & \rightarrow & farm''(\bar{s}_c, \bar{s}_{g}, n, m) \\
    \Gamma, p,c \in \mathbb{F} & \vdash & farm(\bar{s}, n) & \rightarrow & cluster(\bar{s}, p, c) \\
    \begin{array}{r}
      \Gamma, \bar{s}_{g} \in \mathbb{S}, n \in \mathbb{N}^{+}, \\
      x,y \in \mathbb{N}^{0}, m \in \mathbb{R}^{+} 
    \end{array} \bigg\} & \vdash & \bar{f} & \rightarrow & cluster'(\bar{f}, \bar{s}_g, n, m, x, y) \\
    \begin{array}{r} 
      \Gamma, \bar{s}_{g} \in \mathbb{S}, n \in \mathbb{R}^{+}, \\
      g,h,i \in \mathbb{F}
    \end{array} \Bigg\} & \vdash & \bar{f} & \rightarrow & cluster''(\bar{f}, \bar{s}_g, n, g, h, i) \\
  \end{array}
\]

\section{Program Shaping Rewrite Rules}
\label{sec:shaping-rewrites}

\[
  \begin{array}{rclcl}
    \Gamma, c \in \mathbb{N}^{+} & \vdash & \mathrm{skel}(farm(\bar{s}, n), xs) &
                                                                         \rightarrow
    & \mathrm{skel}(cluster(\bar{s}), chunk(xs, c)) \\
    f, xs \in \Gamma, xs \in \mathbbm{L}, f \in \mathbb{F} & \vdash & [f(x)\; ||\; x
                                                                     \leftarrow
                                                                     xs] &
                                                                           \rightarrow &
                                                                           \mathrm{skel}(cluster(\bar{f}),
                                                                           chunk(xs))
    \\
    \Rewrite{x \in \Gamma}{\mathcal{E}_{x\; :\; list\; a}}{\mathcal{E}_{x'\; :\; binary\; a}} 
  \end{array}
\]

\noindent
Where $\mapsto^{\!*}$ represents the composite application of rewrites to the
slice of expressions over $x$, denoted $\mathcal{E}_x$, transforming all relevant list
declarations of $x$ to equivalent binary declarations of $x'$, and all relevant
list operations on $x$ to those equivalent binary operations on $x'$.

\[
  \mathbb{a}
  \mathbb{b}
  \mathbb{c}
  \mathbb{d}
  \mathbb{e}
  \mathbb{f}
  \mathbb{g}
  \mathbb{h}
  \mathbb{i}
  \mathbb{j}
  \mathbb{k}
  \mathbb{l}
  \mathbb{m}
  \mathbb{n}
  \mathbb{o}
  \mathbb{p}
  \mathbb{q}
  \mathbb{r}
  \mathbb{s}
  \mathbb{t}
  \mathbb{u}
  \mathbb{v}
  \mathbb{w}
  \mathbb{x}
  \mathbb{y}
  \mathbb{z}
\]

\section{Text}

All refactorings given in this paper will be described as semi-formal rewrite
rules, operating over the abstract syntax tree (AST) of the source program.
Each refactoring has a set of conditions ensuring the transformation valid, a
description of the syntax to be transformed, and a description of the syntax
following successful transformation. Conditions are given as predicates to each
rule.

Each rewrite rule operates within an environment, $\gamma$, allowing access and
reference to the current scope of the rewrite rule within the source program.
This includes a set of all available functions, $\mathbb{F}$. The skeleton
library \skel, and the skeletons it provides, introduced as part of some below
refactorings, are denoted by the set $\mathbb{S}$.
% 
\[
  \mathbb{S} = \{skel, \bar{f}, pipe, farm, farm', farm'', cluster, cluster',
  cluster''\} 
\]
% 
\noindent
For all rewrite rules, $\mathbb{S}$ is assumed to be in scope; either already
imported, or imported (perhaps by refactoring) for the refactoring itself. To
ensure $\mathbb{S}$ accessible, we extend $\gamma$:
% 
\[
  \Gamma = \gamma \cup \mathbb{S}
\]
% 
\noindent

We next define a series of semantic equivalences to allow for more concise
rewrite rules. Each equivalence is subject to a series of predicates under which
it is valid, and is defined in the form:
% 
\[
  \begin{array}{rclcl}
    \seq{\ing{\s, xs}, \type{xs}{list}{a}}
    {skel(\s, xs)}{\mathtt{skel:do}(\s, xs)} \\
  \end{array}
\]
% 
\noindent
Where $\bar{s}$ represents any valid skeleton in $\mathbb{S}$, \ie{}
$\mathbb{S}/\{skel\}$; and $xs$ evaluates to a list where all elements have the
same type. Semantic equivalences have been defined for each skeleton in \skel;
with all definitions given in Fig.~\ref{fig:rewrite-rules-sem-eqs}.

Using these semantic equivalences we define rewrite rules for each refactoring
introduced here. We divide the rules according to their purpose and type. 



% All refactorings demonstrated in this paper will be described as a semi-formal
% rewrite rule operating over the abstract syntax tree (AST) of the source
% program. Each rewrite rule is predicated by a series of pre-conditions that must
% be met before that transformation can be applied. Pre-conditions are denoted
% using the following definitions.

% \[
%   \begin{array}{rcl}
%     \gamma & = & \text{Program Environment} \\
%     \mathbb{F} & = & \text{Set of all functions in $\gamma$} \\
%     \mathbb{E} & = & \text{Set of all expressions in $\gamma$} \\
%     \mathbbm{e}_f & = & \text{Set of expressions in function $f$, subset of
%                         $\mathbb{E}$} \\
%     \mathbb{L} & = & \text{Set of all possible lists} \\
%     \mathbb{B} & = & \text{Set of all possible binaries} \\
%     \mathbb{S} & = & \{\mathrm{skel}, \bar{f}, pipe, farm, farm', farm'', cluster, cluster',
%                      cluster''\} \\
%     \Gamma & = & \{\gamma \times \mathbb{S}\}
%   \end{array}
% \]




% We next define a series of semantic equivalences to be used in the rewrite
% rules. Each equivalence is predicated by the context under which it is valid,
% using the above definitions. 

% (Section 2 here.)

% \erlang{} allows functions to be defined and referenced in multiple ways. To
% meet these cases, $\mathcal{F}_f$ is defined to represent any valid form of the
% function $f$. Using the above definitions, rewrite rules can be defined. 

% Each rewrite rule is made up of three parts: assumptions, pre-, and post- forms.
% The assumptions can be found on the left of the turnstile, and are made up of a
% series of definitions; primarily expressed in terms of sets. These assumptions
% can mean both the assumption of a working scope, but also of inputs. These
% assumptions are predicates; they predicate the rewrite rule. The rewrite rules
% act over a module/program and apply to all applicable things. These predicates
% are built from a list of assumptions which represent required aspects of the
% module before, and any necessary input to the refactoring from the user. We list
% these assumptions in Section~\ref{}. 

% On the left of the turnstile 

% NB: Mention transitivity of rewrite rules

\end{document}